%%%%%%%%%%%%%%%%%%%%%%%%%%%%%%%%%%%%%%%%%
% University Assignment Title Page 
% LaTeX Template
% Version 1.0 (27/12/12)
%
% This template has been downloaded from:
% http://www.LaTeXTemplates.com
%
% Original author:
% WikiBooks (http://en.wikibooks.org/wiki/LaTeX/Title_Creation)
%
% License:
% CC BY-NC-SA 3.0 (http://creativecommons.org/licenses/by-nc-sa/3.0/)
% 
% Instructions for using this template:
% This title page is capable of being compiled as is. This is not useful for 
% including it in another document. To do this, you have two options: 
%
% 1) Copy/paste everything between \begin{document} and \end{document} 
% starting at \begin{titlepage} and paste this into another LaTeX file where you 
% want your title page.
% OR
% 2) Remove everything outside the \begin{titlepage} and \end{titlepage} and 
% move this file to the same directory as the LaTeX file you wish to add it to. 
% Then add \input{./title_page_1.tex} to your LaTeX file where you want your
% title page.
%
%%%%%%%%%%%%%%%%%%%%%%%%%%%%%%%%%%%%%%%%%
%\title{Title page with logo}
%----------------------------------------------------------------------------------------
%	PACKAGES AND OTHER DOCUMENT CONFIGURATIONS
%----------------------------------------------------------------------------------------
\documentclass[12pt]{article}
\usepackage[utf8]{inputenc}
\usepackage{amsmath}
\usepackage{csquotes}
\usepackage{graphicx}
\usepackage[colorinlistoftodos]{todonotes}
\begin{document}

\begin{titlepage}

\newcommand{\HRule}{\rule{\linewidth}{0.5mm}} % Defines a new command for the horizontal lines, change thickness here

\center % Center everything on the page
 
%----------------------------------------------------------------------------------------
%	HEADING SECTIONS
%----------------------------------------------------------------------------------------
%----------------------------------------------------------------------------------------
%	LOGO SECTION
%----------------------------------------------------------------------------------------
\begin{figure}[b]
\centering
\includegraphics[scale=0.25]{logo.png}\\ % Include a department/university logo - this will require the graphicx package
 \end{figure}
%----------------------------------------------------------------------------------------

\textsc{\LARGE Central Michigan University}\\[1.5cm] % Name of your university/college
\textsc{\Large Department of Computer Science}\\[0.5cm] % Major heading such as course name
\textsc{\large Artificial Intelligence}\\[0.5cm] % Minor heading such as course title

%----------------------------------------------------------------------------------------
%	TITLE SECTION
%----------------------------------------------------------------------------------------

\HRule \\[0.4cm]
{ \huge \bfseries Using Twitter and News Source Sentiment Analysis to Predict Stock Market Movement}\\[0.4cm] % Title of your document
\HRule \\[1.5cm]
 
%----------------------------------------------------------------------------------------
%	AUTHOR SECTION
%----------------------------------------------------------------------------------------

\begin{minipage}{0.4\textwidth}
\begin{flushleft} \large
\emph{Authors:}\\
Hesham \textsc{Salman}\\ % Your name
Neeraj \textsc{Rajesh}\\
Adithi \textsc{Potdar}
\end{flushleft}
\end{minipage}
~
\begin{minipage}{0.4\textwidth}
\begin{flushright} \large
\vspace*{-1.2cm}
\emph{Supervisor:} \\
Dr. Lisa \textsc{Gandy}% Supervisor's Name
\end{flushright}
\end{minipage}\\[2cm]

% If you don't want a supervisor, uncomment the two lines below and remove the section above
%\Large \emph{Author:}\\
%John \textsc{Smith}\\[3cm] % Your name

%----------------------------------------------------------------------------------------
%	DATE SECTION
%----------------------------------------------------------------------------------------

{\large \today}\\[2cm] % Date, change the \today to a set date if you want to be precise
\vfill % Fill the rest of the page with whitespace

\end{titlepage}


\begin{abstract}
As Bollen et al. have shown, Twitter sentiment analysis can be used to predict stock market movement as a whole. Bollen, along with others, have used proprietary algorithms -- such as GPOMS (Google Profile of Mood States) -- to achieve their high model accuracy (87\%). Other researchers have been able to achieve good results with non-proprietary algorithms (60-75\%). By mining \textit{cash}-tags (company trading codes preceded by "\$", frequently used by traders) for \textit{specific} companies and performing a sentiment analysis on the results, we may be able to predict the movement of specific stocks. Sentiment data may have to be normalized; tweet sentiments are cyclical in many ways ("happy" sentiments increase during the holidays, for example), and it stands to reason that financial tweets may be affected by general changes in tweet sentiment. 
\end{abstract}

\section{Previous Work}
In previous analyses, Twitter sentiment was found to precede stock market changes by roughly three days \cite{bollen-ieee}. Self organizing fuzzy neural networks (SOFNN) were used to test the hypotheses, and Granger causality relations were performed to assess the POMS classification bearing on stock market changes \cite{bollen-ieee} \cite{bollen-jcs} \cite{mittal}. Of the POMS classification, only calm was found to be statistically significant \cite{bollen-ieee}. Opinion-finder, a widely used sentiment-analysis tool that classifies text, was lacking in any significant correlation with stock market changes \cite{bollen-ieee}.

\section{Proposed Method}
We propose that a more refined model can be created by mining for specific cash-tags. Furthermore, we can normalize our cash-tag data by mining a subset of Twitter data to analyze the current POMS distribution against the mined cash-tag POMS distribution. We propose sentiment analysis by applying a modified version of Alex Davies' word list \cite{davies}. The collected data must be cleaned, so ideally we will clean it by selecting a subset of the data that is most likely to contain sentimental value. While the Twitter information is being processed, raw Dow-Jones values will be processed and fed into a SOFNN machine learning model to predict future DJIA values and make decisions on a mock portfolio with the goal of maximizing profits. 

\section{Bibliography}
\bibliographystyle{unsrt}%Used BibTeX style is unsrt
\bibliography{proposal}
\end{document}